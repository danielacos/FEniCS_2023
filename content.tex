% DO NOT COMPILE THIS FILE DIRECTLY!
% This is included by the other .tex files.

\begin{frame}[t,plain]
\titlepage
\end{frame}

\begin{frame}{Structure of the talk}
	\tableofcontents
\end{frame}

\section{Tumor model}

% \begin{frame}{Cahn-Hilliard equation}
% 	\begin{block}{}
% 		\vspace*{-0.2cm}
% 		\begin{equation*}
% 			\begin{aligned}
% 				u_t&= \nabla\cdot \left(M(u)\nabla\mu\right)\quad&\text{in }\Omega\times(0,T),\\
% 				\mu&=-\varepsilon^2\Delta u+F'(u)\quad&\text{in }\Omega\times(0,T),\\
% 				\nabla u\cdot \mathbf{n}&=M(u)\nabla\mu\cdot\mathbf{n}=0\quad&\text{on }\partial\Omega\times(0,T),\\
% 				u(0)&=u_0\quad&\text{in }\Omega.
% 			\end{aligned}
% 		\end{equation*}
% 	\end{block}
% 	\begin{itemize}
% 		\item $\alert{F(u)}=\frac{1}{4}u^2(1-u)^2$ Ginzburg-Landau double well functional.
% 		\item $\alert{M(u)}=u_\oplus(1-u)_\oplus$ degenerate mobility function.
% 		\begin{itemize}
% 			\item $v_\oplus \coloneqq\max\{v,0\}$,\quad
% 			$v_\ominus \coloneqq-\min\{v,0\},$\quad
% 			$v=v_\oplus  - v_\ominus$.
% 		\end{itemize}
% 		\item $u$ \structure{minimizes energy functional}:
% 		$$\alert{E(u(t))}\coloneqq\frac{\varepsilon^2}{2}\int_\Omega|\nabla u(t)|^2dx+\int_\Omega F(u(t))dx.$$
% 		\item Pointwise bounds: $u\in[0,1]$. 
% 	\end{itemize}
% \end{frame}

\subsubsection{Tumor model}
\begin{frame}{Tumor model}
	\begin{block}{}
		\begin{center}
			\structure{\textbf{Cahn-Hilliard model (\alert{tumor})}} \alert{+} \structure{\textbf{Diffusion equation (\alert{nutrients})}}
		\end{center}
	\end{block}
	\begin{figure}[h!]
		\centering
		\includegraphics[scale=0.1]{img/celulas.png}
		\caption{Interaction between the species.}
	\end{figure}
	Variables $\alert{u}$ and $\alert{n}$ bounded on $[0,1]$:
	\begin{itemize}
		\item $\alert{u}$ phase-field variable ($1$ tumor cells and $0$ healthy cells).
		\item $\alert{n}$ nutrient-rich extracellular water volume fraction.
	\end{itemize}

	
	%	\structure{\textbf{energía}} $E\colon H^1(\Omega)\times H^1(\Omega)\longrightarrow\R$ se define como 
	%	\begin{align*}
		%	E(u(t),n(t))&\coloneqq\int_\Omega\left(\frac{\varepsilon^2}{2}|\nabla u(x,t)|^2+F(u(x,t))-\chi_0u(x,t)n(x,t)+\frac{1}{2\delta}\left(n(x,t)\right)^2\right)dx.
		%	\end{align*}
\end{frame}

\begin{frame}{Tumor model {\footnotesize (derived from \cite{van_der_zee_model})}}
	\footnotesize
	\vspace*{-0.2cm}
	\begin{block}{}
		\vspace*{-0.4cm}
		\begin{subequations}
			\begin{align*}
				\partial_t u&=C_u\nabla\cdot\left(M(u)\nabla\mu_u\right)+\framedmath<2>{\delta P_0P(u,n)(\mu_n-\mu_u)_\oplus} \quad&\text{in }\Omega\times (0,T),\\
				\mu_u&=F'(u)-\varepsilon^2\Delta u\framedmath<3>{-\chi_0 n}\quad&\text{in }\Omega\times (0,T),\\
				\partial_t n&=C_n\nabla\cdot\left(M(n)\nabla\mu_n\right)-\framedmath<2>{\delta P_0P(u,n)(\mu_n-\mu_u)_\oplus} \quad&\text{in }\Omega\times (0,T),\\
				\mu_n&=\frac{1}{\delta} n \framedmath<3>{-\chi_0 u} \quad&\text{in }\Omega\times (0,T),\\
				\nabla u\cdot \mathbf{n}&=\left( M(n)\nabla \mu_n\right)\cdot \mathbf{n}=\left( M(u)\nabla \mu_u\right)\cdot \mathbf{n}=0 \quad &\text{on }\partial\Omega\times (0,T),\\
				u(0)&=u_0,\quad n(0)=n_0\quad&\text{in }\Omega,%\\
				%\mu_u(0)&=f'(u_0)-\varepsilon^2\Delta u_0-\chi_0 n_0\quad&\text{en }\Omega,
			\end{align*}
		\end{subequations}
	\end{block}
	where $u_0,n_0\in L^2(\Omega)$are the initial conditions.
	
	\vspace*{0.2cm}
	\begin{itemize}
		\item $\alert{M(v)}\coloneqq h_{p,q}(v)$, $\alert{P(u,n)}\coloneqq h_{r,s}(u)n_\oplus$ for certain $p,q,r,s\in\N$, where 
		$$
		\alert{h_{p,q}(v)}\coloneqq K_{p,q}v_\oplus^p(1-v)_\oplus^q
		$$
		with $K_{p,q}>0$ a constant so that $\max_{x\in\R}h_{p,q}(v)=1$.
		\item<2> \myframed{Cells-nutrients chemical reactions.}
		\item<3> \myframed{Movement of tumor cells $\longleftarrow$ \textbf{Cross-diffusion terms}.}
	\end{itemize}	
\end{frame}

\begin{frame}{Properties of the model}
	\begin{proposition}[Mass conservation]
		The total mass of tumor cells and nutrients is conserved: $$\alert{\frac{d}{dt}\int_\Omega (u(x,t)+n(x,t))dx=0}.$$
	\end{proposition}
	\vspace*{0.6cm}
	\begin{proposition}[Pointwise bounds]
		Let $u_0,v_0\in[0,1]$, then \alert{$u(t),n(t)\in[0,1]$} for a.e. $t\in(0,T)$.
	\end{proposition}
\end{frame}
\begin{frame}{Properties of the model}
	\scriptsize
	\begin{proposition}[Energy law]
		If $\partial_t u\in L^2(0,T, H^1(\Omega))$, then it satisfies the following energy law
		\begin{align*}
			\frac{d E(u(t),n(t))}{dt}&+C_u\int_\Omega M(u(x,t))|\nabla\mu_u(x,t)|^2dx+C_n\int_\Omega M(n(x,t))|\nabla\mu_n(x,t)|^2\notag\\&+\delta P_0\int_\Omega P(u(x,t),n(x,t))(\mu_u(x,t)-\mu_n(x,t))_\oplus ^2dx=0,
		\end{align*}
		where
		\begin{align*}
			\alert{E(u(t),n(t))}&\coloneqq\int_\Omega\left(\frac{\varepsilon^2}{2}|\nabla u(x,t)|^2+ F(u(x,t))-\chi_0u(x,t)n(x,t)+\frac{1}{2\delta}\left(n(x,t)\right)^2\right)dx.
		\end{align*}
		Therefore, the solution is \structure{energy stable} in the sense $$\alert{\frac{d}{dt}E(u(t),n(t))\le 0}.$$
	\end{proposition}
\end{frame}

\section{Approximation of the model}
\begin{frame}{FE and DG methods}
	\footnotesize
	\vspace*{-0.2cm}
	\begin{block}{}
		\text{Finite Element method:}
		\begin{equation*}
			\alert{\Pc_k(\T_h)}\coloneqq\left\{v_h\in \mathcal{C}^0(\overline\Omega)\colon v_{h|_{K_i}}\in\mathbb{P}_k(K_i)\text{ with } K_i\in\T_h,\forall i\in\left\{1,2,\ldots,N_{\T_h}\right\}\right\},
		\end{equation*}
		\text{Discontinuous Galerkin method:}
		\begin{equation*}
			\alert{\Pd_k(\T_h)}\coloneqq\left\{v_h\in L^2(\Omega)\colon v_{h|_{K_i}}\in\mathbb{P}_k(K_i)\text{ with } K_i\in\T_h,\forall i\in\left\{1,2,\ldots,N_{\T_h}\right\}\right\}.
		\end{equation*}
	\end{block}
	% with a basis $\left\{\phi_i\right\}_{i\in\left\{1,2,\ldots,N_h \right\}}$.
	
	\vspace*{0.3cm}
	Notation:
	
	\begin{minipage}{0.69\textwidth}
	\begin{itemize}
		\item \structure{Average}: 
		$\alert{\media{v}}\coloneqq
		\begin{cases}
			\dfrac{\vK+\vL}{2}&\text{if } e=\partial K\cap\partial L\in\E_h^i\\
			\vK&\text{if }e=\partial K\in\E_h^b
		\end{cases},$
		\item \structure{Jump}: $
		\alert{\salto{v}}\coloneqq
		\begin{cases}
			\vK-\vL&\text{if } e=\partial K\cap\partial L\in\E_h^i\\
			\vK&\text{if }e=\partial K\in\E_h^b
		\end{cases},$
	\end{itemize}
	\end{minipage}
	\begin{minipage}{0.29\textwidth}
		\vspace*{-0.5cm}
		\begin{figure}
			\centering
			\includegraphics[scale=0.7]{img/orientation_n.pdf}
			{\scriptsize\structure{Figure:} Orientation of unit normal vector.}
		\end{figure}
	\end{minipage}
\end{frame}

\begin{frame}{Projection and regularization operators}
	Let $g\in L^1(\Omega)$.
	\vspace*{0.3cm}

	\begin{itemize}
		\item \structure{Projection} $\alert{\Pi_0}\colon L^1(\Omega) \rightarrow \Pd_0(\T_h)$:
		\begin{equation*}
			\escalarL{g}{\overline{w}}=
			\escalarL{\alert{\Pi_0 g}}{\overline{w}},\forall\,\overline{w}\in \Pd_0(\T_h).
		\end{equation*}
		\vspace*{0.1cm}
		\item \structure{Regularization} $\alert{\Pi^h_1}\colon L^1(\Omega)\rightarrow \Pc_1(\T_h)$
		\begin{equation*}
			\escalarL{g}{\overline{\phi}}=\escalarML{\alert{\Pi^h_1 g}}{\overline{\phi}},\forall\,\overline{\phi}\in \Pc_1(\T_h),
		\end{equation*}
		where $\alert{\escalarML{\cdot}{\cdot}}$ is the \structure{mass-lumping scalar product} in $\Pc_1(\T_h)$.
	\end{itemize}
\end{frame}

\begin{frame}{Mesh assumptions}
	\begin{hypothesis}
		\label{hyp:mesh_n}
		The mesh $\T_h$ of $\overline\Omega$ is structured in the sense that the line between the baricenters of the triangles $K$ and $L$ is orthogonal to the interface $e=K\cap L\in\E_h^i$.
	\end{hypothesis}
	\vspace*{0.5cm}
	\begin{figure}
		\centering
		\includegraphics[scale=0.55]{img/adjacent_baricenters.pdf}
		\caption{Polygonal structure between adjacent baricenters.}
	\end{figure}
\end{frame}
\begin{frame}{Mesh assumptions}
	With this assumption:
	$$\nabla\mu \cdot \nn_e \simeq\frac{-\salto{\Pi_0\mu}}{\mathcal{D}_e(\T_h)}= \frac{\Pi_0\mu_L-\Pi_0\mu_K}{\mathcal{D}_e(\T_h)}\eqqcolon \alert{\nabla_{\nn_e}^0\mu},$$
	on every $e=K\cap L\in\Ehi$ with $\alert{\mathcal{D}_e(\T_h)}$ the distance between the baricenters of the triangles $K$ and $L$.

	\vspace*{0.3cm}
	\begin{block}{}
		For $\mu\in\Pd_1(\T_h)$, $\nabla_{\nn_e}^0\mu$ is the slope of the line between $(C_K, \mu(C_K))$ and $(C_L, \mu(C_L))$, with $C_K, C_L$ the baricenters of $K, L\in\T_h$.
	\end{block}
	\vspace*{0.3cm}
	More information in \cite{acosta-soba_KS_2022}.
\end{frame}

\begin{frame}{Upwind method {\footnotesize (derived from \cite{acosta-soba_CH_2022})}}
	\footnotesize
	Notice that
	$$\nabla\cdot(M(v)\nabla\mu)=M'(v)\nabla\mu \cdot\nabla v+M(v)\Delta\mu.$$
	Hence, $\alert{M'(v)}$ determines the direction of the flux.
	
	\vspace*{0.3cm}
	% \begin{itemize}
	% 	\item If $u\in[0,1]$ then $M(u)=M(u)_\oplus$.
	% \end{itemize}
	% \vspace*{0.3cm}
	
	Consider $M(v)=M^\uparrow(v)+M^\downarrow(v)$:
	\begin{itemize}
		\item Increasing part of $M(v)$: $\alert{M^\uparrow(v)}=
		\begin{cases}
			M(v) & \text{if }v\le \frac{1}{2}\\[0.2em]
			M\left(\frac{1}{2}\right) & \text{if } v>\frac{1}{2}
		\end{cases}.$
		\item Decreasing part of $M(v)$: $
		\alert{M^\downarrow(v)}=
		\begin{cases}
			0 & \text{if }v\le \frac{1}{2}\\
			M(v)-M\left(\frac{1}{2}\right) & \text{if } v>\frac{1}{2}
		\end{cases}.$
	\end{itemize}

	% \vspace*{0.3cm}
	\begin{block}{}
	$a_h^{\text{upw}}:\Pd_k(\T_h)\times \Pd_k(\T_h)\to\R$,
	\scriptsize
	\begin{multline*}
		\alert{\aupw{\mu}{M(v)}{\bv}}\coloneqq-\int_\Omega (\nabla\mu\cdot\nabla\overline{v})M(v)\\+\sum_{e\in\E_h^i,e=K\cap L}\int_e\left((\nabla_{\nn_e}^0\mu)_{\oplus}(M^\uparrow(\vK)+M^\downarrow(\vL))_\oplus-(\nabla_{\nn_e}^0\mu)_{\ominus}(M^\uparrow(\vL)+M^\downarrow(\vK))_\oplus\right)\salto{\bv}
	\end{multline*}
	\end{block}
\end{frame}

\begin{frame}{Numerical scheme}
	\footnotesize
	Given $u^m,n^m\in\Pd_0(\T_h)$ with $u^{m},n^{m}\in[0,1]$ and $\mu_u^m\in\Pc_1(\T_h)$, find $u^{m+1}, n^{m+1}\in \Pd_0(\T_h)$ and $\mu_u^{m+1} \in \Pc_1(\T_h)$, such that
	\begin{block}{}
		\begin{align*}
			\escalarL{\delta_tu^{m+1}}{\overline{u}}&=-C_u\aupw{\mu_u^{m+1}}{M(u^{m+1})}{\overline{u}}&\notag\\&\quad+\delta P_0\escalarL{P(u^{m+1},n^{m+1})(\mu_n^{m+1}-\Pi_0\mu_u^{m+1})_\oplus }{\overline{u}},\\
			\escalarML{\mu_u^{m+1}}{\overline{\mu}_u}&=\varepsilon^2 \escalarLd{\nabla \up^{m+1}}{\nabla\overline{\mu}_u}+ \escalarL{f(\up^{m+1},\up^m)}{\overline{\mu}_u}\nonumber\\&\quad-\chi_0\escalarL{ n^{m+1}}{\overline{\mu}_u},\\
			\escalarL{\delta_t n^{m+1}}{\overline{n}}&=-C_n\aupw{\mu_n^{m+1}}{M(n^{m+1})}{\overline{n}}&\notag\\&\quad-\delta P_0 \escalarL{P(u^{m+1},n^{m+1})(\mu_n^{m+1}-\Pi_0\mu_u^{m+1})_\oplus }{\overline{n}},\\
			\mu_n^{m+1}&=\frac{1}{\delta}n^{m+1}  -\chi_0 \Pi_0(\up^{m}),
		\end{align*}
	\end{block}
	$\forall\overline{u},\overline{n}\in\Pd_0(\T_h)$, $\forall\overline{\mu}_u\in \Pc_1(\T_h)$, where $u^0=u_0$, $n^0=n_0$
\end{frame}

\begin{frame}{Properties of the discrete scheme}
	\begin{proposition}[Mass conservation]
		The total mass of cells and nutrients is conserved: for all $m\ge 0$,
		\footnotesize
		\begin{align*}
		\alert{\int_\Omega (u^{m+1}+n^{m+1})=\int_\Omega (u^m+n^m)}\quad\text{and}\quad\alert{\int_\Omega (\up^{m+1}+n^{m+1})=\int_\Omega (\up^m+n^m)}.
		\end{align*}
	\end{proposition}
	\vspace*{0.3cm}
	\begin{theorem}[Pointwise bounds]
		Let $u^{m},n^{m}\in[0,1]$, then $\alert{u^{m+1},\up^{m+1},n^{m+1}\in[0,1]}$.
	\end{theorem}
	\vspace*{0.3cm}
	\begin{theorem}[Existence of solution]
		There is at least one solution of the scheme.
	\end{theorem}
\end{frame}
\begin{frame}{Properties of the discrete scheme}
	\begin{theorem}[Energy law]
		Any solution satisfies the following discrete energy law
		{\footnotesize
		\begin{align*}
			\delta_t E(\up^{m+1},n^{m+1})&+C_u\aupw{\mup_u^{m+1}}{M(u^{m+1})}{\Pi_0\mu_u^{m+1}}\\&+C_n\aupw{\mu_n^{m+1}}{M(n^{m+1})}{\mu_n^{m+1}}\notag\\&+\frac{\Delta t\varepsilon^2}{2}\int_\Omega|\delta_t\nabla \up^{m+1}|^2+\frac{\Delta t}{2\delta}\int_\Omega \vert\delta_t n^{m+1}\vert^2\nonumber\\&+\delta P_0\int_\Omega P(u^{m+1},n^{m+1}) (\mu_n^{m+1}-\mup_u^{m+1})_\oplus ^2
			\le 0.
		\end{align*}
		}
	\end{theorem}
	\begin{corollary}[Energy stability]
		The scheme is \structure{unconditionally energy stable} in the sense
		$$
		\alert{E(\up^{m+1},n^{m+1})\le E(\up^{m},n^{m})}.
		$$
	\end{corollary}
\end{frame}

\section{Numerical tests}

\subsection{Three tumors aggregation}

\begin{frame}{Three tumors aggregation}
	% {\tiny
	% \begin{align*}
	% 	u_0 &= \frac{1}{2}\left[\tanh\left(\frac{1 - \sqrt{(x- 2)^2 + (y - 2)^2}}{\sqrt{2}\varepsilon}\right)
	% 	+ \tanh\left(\frac{1 - \sqrt{(x - 3)^2 + (y + 5)^2}}{\sqrt{2}\varepsilon}\right)\right.\\&\quad\left.
	% 	+ \tanh\left(\frac{1.73 - \sqrt{(x + 1.5)^2 + (y + 1.5)^2}}{\sqrt{2}\varepsilon}\right) + 3\right],\\
	% 	n_0 &= 1.0 - u_0.
	% \end{align*}}

	\begin{figure}
		\centering
		\begin{tabular}{cc}
			\hspace*{-0.9cm}$\boldsymbol{u_0}$ & \hspace*{-0.9cm}$\boldsymbol{n_0}$\\
			\includegraphics[scale=0.2]{img/three_tumors/initial_cond/tumor_DG-UPW_Pi1_u_i-0_cropped.png} &
			\includegraphics[scale=0.2]{img/three_tumors/initial_cond/tumor_DG-UPW_Pi1_n_i-0_cropped.png}
		\end{tabular}
		% \caption{Initial conditions for test \ref{sec:numer-experiments_1} ($u_0$ left, $n_0$ right).}
	\end{figure}

	Symmetric mobility and proliferation functions:
	$$
	M(v)=h_{1,1}(v),\quad P(u,n)=h_{1,1}(u)n_\oplus.
	$$

	Parameters: $C_u=100$, $C_n=100\cdot 10^{-4}$, $P_0=125$,  $h\approx 0.14$.
\end{frame}

% \begin{frame}{Three tumors aggregation ($\chi_0=0$, $\Delta t=10^{-5}$)}
% 	\scriptsize
% 	\begin{figure}
% 		\centering
% 		\hspace*{-0.6cm}
% 		\begin{tabular}{cccc}
% 			&\hspace*{-1cm}$t=2.5\cdot 10^{-2}$ & \hspace*{-1cm}$t=5.0\cdot 10^{-2}$ & \hspace*{-1cm}$t=7.5\cdot 10^{-2}$ \\
% 			\rotatebox[origin=c]{90}{\textbf{DG}} &
% 			\raisebox{-0.47\height}{\includegraphics[scale=0.145]{img/three_tumors/test_DG_P0-125_dt-1e-5_nx-200_symmetric/tumor_DG-UPW_Pi1_u_i-2500_cropped.png}} &
% 			\raisebox{-0.47\height}{\includegraphics[scale=0.145]{img/three_tumors/test_DG_P0-125_dt-1e-5_nx-200_symmetric/tumor_DG-UPW_Pi1_u_i-5000_cropped.png}} &
% 			\raisebox{-0.47\height}{\includegraphics[scale=0.145]{img/three_tumors/test_DG_P0-125_dt-1e-5_nx-200_symmetric/tumor_DG-UPW_Pi1_u_i-7500_cropped.png}} \\
% 			\rotatebox[origin=c]{90}{\textbf{FE}} &
% 			\raisebox{-0.47\height}{\includegraphics[scale=0.145]{img/three_tumors/test_FEM_P0-125_dt-1e-5_nx-200_symmetric/tumor_FEM_u_i-2500_cropped.png}} &
% 			\raisebox{-0.47\height}{\includegraphics[scale=0.145]{img/three_tumors/test_FEM_P0-125_dt-1e-5_nx-200_symmetric/tumor_FEM_u_i-5000_cropped.png}} &
% 			\raisebox{-0.47\height}{\includegraphics[scale=0.145]{img/three_tumors/test_FEM_P0-125_dt-1e-5_nx-200_symmetric/tumor_FEM_u_i-7500_cropped.png}}
% 		\end{tabular}
% 		\caption{Tumor cells.}
% 	\end{figure}
% \end{frame}
\begin{frame}{Three tumors aggregation ($\chi_0=0$, $\Delta t=10^{-5}$)}
	\scriptsize
	\begin{figure}
		\centering
		\hspace*{-0.6cm}
		\begin{tabular}{cc}
		\hspace*{-1cm} DG & \hspace*{-1cm} FE \\
		\animategraphics[autoplay,loop,width=5cm]{5}{img/animation/three_tumors/test_DG_P0-125_dt-1e-5_nx-200_symmetric/tumor/tumor_DG-UPW_Pi1_u_i_cropped-}{0}{75} &
		\animategraphics[autoplay,loop,width=5cm]{5}{img/animation/three_tumors/test_FEM_P0-125_dt-1e-5_nx-200_symmetric/tumor/tumor_FEM_u_i_cropped-}{0}{75}
		% \includemovie[autoplay]{5cm}{4.05cm}{img/three_tumors/test_DG_P0-125_dt-1e-5_nx-200_symmetric/tumor_DG-UPW_Pi1_u_reduced.mp4} &
		% \includemovie[autoplay]{5cm}{4.05cm}{img/three_tumors/test_FEM_P0-125_dt-1e-5_nx-200_symmetric/tumor_FEM_u_reduced.mp4}
		% \movie{}{img/three_tumors/test_DG_P0-125_dt-1e-5_nx-200_symmetric/tumor_DG-UPW_Pi1_u.gif} &
		% \movie{}{img/three_tumors/test_FEM_P0-125_dt-1e-5_nx-200_symmetric/tumor_FEM_u.gif}
		\end{tabular}
		\caption{Tumor cells.}
	\end{figure}
\end{frame}
% \begin{frame}{Three tumors aggregation ($\chi_0=0$, $\Delta t=10^{-5}$)}
% 	\scriptsize
% 	\begin{figure}
% 		\centering
% 		\hspace*{-0.6cm}
% 		\begin{tabular}{cccc}
% 			&\hspace*{-1cm}$t=2.5\cdot 10^{-2}$ & \hspace*{-1cm}$t=5.0\cdot 10^{-2}$ & \hspace*{-1cm}$t=7.5\cdot 10^{-2}$ \\
% 			\rotatebox[origin=c]{90}{\textbf{DG}} &
% 			\raisebox{-0.47\height}{\includegraphics[scale=0.145]{img/three_tumors/test_DG_P0-125_dt-1e-5_nx-200_symmetric/tumor_DG-UPW_Pi1_n_i-2500_cropped.png}} &
% 			\raisebox{-0.47\height}{\includegraphics[scale=0.145]{img/three_tumors/test_DG_P0-125_dt-1e-5_nx-200_symmetric/tumor_DG-UPW_Pi1_n_i-5000_cropped.png}} &
% 			\raisebox{-0.47\height}{\includegraphics[scale=0.145]{img/three_tumors/test_DG_P0-125_dt-1e-5_nx-200_symmetric/tumor_DG-UPW_Pi1_n_i-7500_cropped.png}} \\
% 			\rotatebox[origin=c]{90}{\textbf{FE}} &
% 			\raisebox{-0.47\height}{\includegraphics[scale=0.145]{img/three_tumors/test_FEM_P0-125_dt-1e-5_nx-200_symmetric/tumor_FEM_n_i-2500_cropped.png}} &
% 			\raisebox{-0.47\height}{\includegraphics[scale=0.145]{img/three_tumors/test_FEM_P0-125_dt-1e-5_nx-200_symmetric/tumor_FEM_n_i-5000_cropped.png}} &
% 			\raisebox{-0.47\height}{\includegraphics[scale=0.145]{img/three_tumors/test_FEM_P0-125_dt-1e-5_nx-200_symmetric/tumor_FEM_n_i-7500_cropped.png}}
% 		\end{tabular}
% 		\caption{Nutrients.}
% 		% \label{fig:test-1_1}
% 	\end{figure}
% \end{frame}
% \begin{frame}{Three tumors aggregation ($\chi_0=0$, $\Delta t=10^{-5}$)}
% 	\begin{figure}
% 		\centering
% 		\begin{tabular}{cc}
% 			\hspace*{1cm}$\boldsymbol{u}$ & \hspace*{0.5cm}$\boldsymbol{n}$\\
% 			\includegraphics[scale=0.33]{img/three_tumors/tumor_min-max_u_chi-0.png} &
% 			\includegraphics[scale=0.33]{img/three_tumors/tumor_min-max_n_chi-0.png}
% 		\end{tabular}
% 		\caption{Pointwise bounds.}
% 	\end{figure}
% \end{frame}
% \begin{frame}{Three tumors aggregation ($\chi_0=0$, $\Delta t=10^{-5}$)}
% 	\begin{figure}
% 		\centering
% 		\includegraphics[scale=0.45]{img/three_tumors/tumor_energy_chi-0.png}
% 		\caption{Energy.}
% 	\end{figure}
% \end{frame}

% \begin{frame}{Three tumors aggregation ($\chi_0=10$, $\Delta t=6\cdot 10^{-5}$)}
% 	\scriptsize
% 	\begin{figure}
% 		\centering
% 		\hspace*{-0.6cm}
% 		\begin{tabular}{cccc}
% 			& \hspace*{-1cm}$t=7.5\cdot 10^{-3}$ & \hspace*{-1cm}$t=1.5\cdot 10^{-2}$ & \hspace*{-1cm}$t=4\cdot 10^{-2}$ \\
% 			\rotatebox[origin=c]{90}{\textbf{DG}} &
% 			\raisebox{-0.47\height}{\includegraphics[scale=0.145]{img/three_tumors/test_DG_P0-125_chi-10_dt-5e-6_nx-200_symmetric/tumor_DG-UPW_Pi1_u_i-1500_cropped.png}} &
% 			\raisebox{-0.47\height}{\includegraphics[scale=0.145]{img/three_tumors/test_DG_P0-125_chi-10_dt-5e-6_nx-200_symmetric/tumor_DG-UPW_Pi1_u_i-3000_cropped.png}} &
% 			\raisebox{-0.47\height}{\includegraphics[scale=0.145]{img/three_tumors/test_DG_P0-125_chi-10_dt-5e-6_nx-200_symmetric/tumor_DG-UPW_Pi1_u_i-8000_cropped.png}} \\
% 			\rotatebox[origin=c]{90}{\textbf{FE}} &
% 			\raisebox{-0.47\height}{\includegraphics[scale=0.145]{img/three_tumors/test_FEM_P0-125_chi-10_dt-5e-6_nx-200_symmetric/tumor_FEM_u_i-1500_cropped.png}} &
% 			\raisebox{-0.47\height}{\includegraphics[scale=0.145]{img/three_tumors/test_FEM_P0-125_chi-10_dt-5e-6_nx-200_symmetric/tumor_FEM_u_i-3000_cropped.png}} &
% 			\raisebox{-0.47\height}{\includegraphics[scale=0.145]{img/three_tumors/test_FEM_P0-125_chi-10_dt-5e-6_nx-200_symmetric/tumor_FEM_u_i-8000_cropped.png}}
% 		\end{tabular}
% 		\caption{Tumor cells.}
% 	\end{figure}
% \end{frame}
\begin{frame}{Three tumors aggregation ($\chi_0=10$, $\Delta t=6\cdot 10^{-5}$)}
	\scriptsize
	\begin{figure}
		\centering
		\hspace*{-0.6cm}
		\begin{tabular}{cc}
			\hspace*{-1cm} DG & \hspace*{-1cm} FE \\
			\animategraphics[autoplay,loop,width=5cm]{5}{img/animation/three_tumors/test_DG_P0-125_chi-10_dt-5e-6_nx-200_symmetric/tumor/tumor_DG-UPW_Pi1_u_i_cropped-}{0}{80} &
			\animategraphics[autoplay,loop,width=5cm]{5}{img/animation/three_tumors/test_FEM_P0-125_chi-10_dt-5e-6_nx-200_symmetric/tumor/tumor_FEM_u_i_cropped-}{0}{80}
			% \includemovie[autoplay]{5cm}{4.05cm}{img/three_tumors/test_DG_P0-125_chi-10_dt-5e-6_nx-200_symmetric/tumor_DG-UPW_Pi1_u_reduced.mp4} &
			% \includemovie[autoplay]{5cm}{4.05cm}{img/three_tumors/test_FEM_P0-125_chi-10_dt-5e-6_nx-200_symmetric/tumor_FEM_u_reduced.mp4}
		\end{tabular}
		\caption{Tumor cells.}
	\end{figure}
\end{frame}
% \begin{frame}{Three tumors aggregation ($\chi_0=10$, $\Delta t=6\cdot 10^{-5}$)}
% 	\scriptsize
% 	\begin{figure}
% 		\centering
% 		\hspace*{-0.6cm}
% 		\begin{tabular}{cccc}
% 			& \hspace*{-1cm}$t=7.5\cdot 10^{-3}$ & \hspace*{-1cm}$t=1.5\cdot 10^{-2}$ & \hspace*{-1cm}$t=4\cdot 10^{-2}$ \\
% 			\rotatebox[origin=c]{90}{\textbf{DG}} &
% 			\raisebox{-0.47\height}{\includegraphics[scale=0.145]{img/three_tumors/test_DG_P0-125_chi-10_dt-5e-6_nx-200_symmetric/tumor_DG-UPW_Pi1_n_i-1500_cropped.png}} &
% 			\raisebox{-0.47\height}{\includegraphics[scale=0.145]{img/three_tumors/test_DG_P0-125_chi-10_dt-5e-6_nx-200_symmetric/tumor_DG-UPW_Pi1_n_i-3000_cropped.png}} &
% 			\raisebox{-0.47\height}{\includegraphics[scale=0.145]{img/three_tumors/test_DG_P0-125_chi-10_dt-5e-6_nx-200_symmetric/tumor_DG-UPW_Pi1_n_i-8000_cropped.png}} \\
% 			\rotatebox[origin=c]{90}{\textbf{FE}} &
% 			\raisebox{-0.47\height}{\includegraphics[scale=0.145]{img/three_tumors/test_FEM_P0-125_chi-10_dt-5e-6_nx-200_symmetric/tumor_FEM_n_i-1500_cropped.png}} &
% 			\raisebox{-0.47\height}{\includegraphics[scale=0.145]{img/three_tumors/test_FEM_P0-125_chi-10_dt-5e-6_nx-200_symmetric/tumor_FEM_n_i-3000_cropped.png}} &
% 			\raisebox{-0.47\height}{\includegraphics[scale=0.145]{img/three_tumors/test_FEM_P0-125_chi-10_dt-5e-6_nx-200_symmetric/tumor_FEM_n_i-8000_cropped.png}}
% 		\end{tabular}
% 		\caption{Nutrients.}
% 	\end{figure}
% \end{frame}
\begin{frame}{Three tumors aggregation ($\chi_0=10$, $\Delta t=6\cdot 10^{-5}$)}
	\scriptsize
	\begin{figure}
		\centering
		\hspace*{-0.6cm}
		\begin{tabular}{cc}
			\hspace*{-1cm} DG & \hspace*{-1cm} FE \\
			\animategraphics[autoplay,loop,width=5cm]{5}{img/animation/three_tumors/test_DG_P0-125_chi-10_dt-5e-6_nx-200_symmetric/nutrients/tumor_DG-UPW_Pi1_n_i_cropped-}{0}{80}
			&
			\animategraphics[autoplay,loop,width=5cm]{5}{img/animation/three_tumors/test_FEM_P0-125_chi-10_dt-5e-6_nx-200_symmetric/nutrients/tumor_FEM_n_i_cropped-}{0}{80}
			% \includemovie[repeat,autoplay]{5cm}{4.05cm}{img/three_tumors/test_DG_P0-125_chi-10_dt-5e-6_nx-200_symmetric/tumor_DG-UPW_Pi1_n_reduced.mp4}  &
			% \includemovie[repeat,autoplay]{5cm}{4.05cm}{img/three_tumors/test_FEM_P0-125_chi-10_dt-5e-6_nx-200_symmetric/tumor_FEM_n_reduced.mp4}
		\end{tabular}
		\caption{Nutrients.}
	\end{figure}
\end{frame}
\begin{frame}{Three tumors aggregation ($\chi_0=0$, $\Delta t=10^{-5}$)}
	\begin{figure}
		\centering
		\begin{tabular}{cc}
			\hspace*{1cm}$\boldsymbol{u}$ & \hspace*{0.5cm}$\boldsymbol{n}$\\
			\includegraphics[scale=0.33]{img/three_tumors/tumor_min-max_u_chi-10.png} &
			\includegraphics[scale=0.33]{img/three_tumors/tumor_min-max_n_chi-10.png}
		\end{tabular}
		\caption{Pointwise bounds.}
	\end{figure}
\end{frame}
\begin{frame}{Three tumors aggregation ($\chi_0=0$, $\Delta t=10^{-5}$)}
	\begin{figure}
		\centering
		\includegraphics[scale=0.45]{img/three_tumors/tumor_energy_chi-10.png}
		\caption{Energy.}
	\end{figure}
\end{frame}

\subsection{Irregular tumor growth}

\begin{frame}{Irregular tumor growth {\footnotesize (more tests in \cite{acosta2023structure})}}
	\footnotesize
	\begin{figure}
		\centering
		\begin{tabular}{cc}
			\hspace*{-1.1cm}$\boldsymbol{u_0}$ & \hspace*{-1.1cm}$\boldsymbol{n_0}$\\
			\includegraphics[scale=0.2]{img/irregular_shape/initial_cond/tumor_DG-UPW_Pi1_u_i-0_cropped.png} &
			\includegraphics[scale=0.2]{img/irregular_shape/initial_cond/tumor_DG-UPW_Pi1_n_i-0_cropped.png}
		\end{tabular}
	\end{figure}

	Symmetric mobility and proliferation functions:
	$$
	M(v)=h_{1,1}(v),\quad P(u,n)=h_{1,1}(u)n_\oplus.
	$$
	Nonsymmetric mobility and proliferation functions:
	$$
	M(v)=h_{5,1}(v),\quad P(u,n)=h_{1,3}(u) n_\oplus.
	$$

	Parameters: $C_u=2.8$, $C_n=2.8\cdot 10^{-4}$, $h\approx 0.28$, $P_0=0.5$, $\chi_0=0.1$, $\Delta t=0.1$.
\end{frame}
% \begin{frame}{Irregular tumor growth {\footnotesize (more tests in \cite{acosta2023structure})}}
% 	\scriptsize
% 	\begin{figure}
% 		\centering
% 		\hspace*{-0.6cm}
% 		\begin{tabular}{ccccc}
% 			& \hspace*{-1cm}$t=10.0$ & \hspace*{-1cm}$t=20.0$ & \hspace*{-1cm}$t=50.0$ \\
% 			\rotatebox[origin=c]{90}{\textbf{Symmetric}} &
% 			\raisebox{-0.47\height}{\includegraphics[scale=0.145]{img/irregular_shape/reference_test_symmetric/tumor_DG-UPW_Pi1_u_i-100_cropped.png}} &
% 			\raisebox{-0.47\height}{\includegraphics[scale=0.145]{img/irregular_shape/reference_test_symmetric/tumor_DG-UPW_Pi1_u_i-200_cropped.png}} &
% 			\raisebox{-0.47\height}{\includegraphics[scale=0.145]{img/irregular_shape/reference_test_symmetric/tumor_DG-UPW_Pi1_u_i-500_cropped.png}} \\
% 			\rotatebox[origin=c]{90}{\textbf{Non-symmetric}} &
% 			\raisebox{-0.47\height}{\includegraphics[scale=0.145]{img/irregular_shape/reference_test/tumor_DG-UPW_Pi1_u_i-100_cropped.png}} &
% 			\raisebox{-0.47\height}{\includegraphics[scale=0.145]{img/irregular_shape/reference_test/tumor_DG-UPW_Pi1_u_i-200_cropped.png}} &
% 			\raisebox{-0.47\height}{\includegraphics[scale=0.145]{img/irregular_shape/reference_test/tumor_DG-UPW_Pi1_u_i-500_cropped.png}}
% 		\end{tabular}
% 		\caption{Tumor cells.}
% 	\end{figure}
% \end{frame}
\begin{frame}{Irregular tumor growth {\footnotesize (more tests in \cite{acosta2023structure})}}
	\scriptsize
	\begin{figure}
		\centering
		\hspace*{-0.6cm}
		\begin{tabular}{cc}
			\hspace*{-1cm} Symmetric & \hspace*{-1cm} Non-symmetric \\
			\animategraphics[autoplay,loop,width=5cm]{5}{img/animation/irregular_shape/reference_test_symmetric/tumor/tumor_DG-UPW_Pi1_u_i_cropped-}{0}{50} &
			\animategraphics[autoplay,loop,width=5cm]{5}{img/animation/irregular_shape/reference_test/tumor/tumor_DG-UPW_Pi1_u_i_cropped-}{0}{50}

			% \includemovie[autoplay]{5cm}{4.05cm}{img/irregular_shape/reference_test_symmetric/tumor_DG-UPW_Pi1_u_reduced.mp4} &
			% \includemovie[autoplay]{5cm}{4.05cm}{img/irregular_shape/reference_test/tumor_DG-UPW_Pi1_u_reduced.mp4}
		\end{tabular}
		\caption{Tumor cells.}
		\label{fig:test-2_reference}
	\end{figure}
\end{frame}
% \begin{frame}{Irregular tumor growth {\footnotesize (more tests in \cite{acosta2023structure})}}
% 	\scriptsize
% 	\begin{figure}
% 		\centering
% 		\hspace*{-0.6cm}
% 		\begin{tabular}{ccccc}
% 			& \hspace*{-1cm}$t=10.0$ & \hspace*{-1cm}$t=20.0$ & \hspace*{-1cm}$t=50.0$ \\
% 			\rotatebox[origin=c]{90}{\textbf{Symmetric}} &
% 			\raisebox{-0.47\height}{\includegraphics[scale=0.145]{img/irregular_shape/reference_test_symmetric/tumor_DG-UPW_Pi1_n_i-100_cropped.png}} &
% 			\raisebox{-0.47\height}{\includegraphics[scale=0.145]{img/irregular_shape/reference_test_symmetric/tumor_DG-UPW_Pi1_n_i-200_cropped.png}} &
% 			\raisebox{-0.47\height}{\includegraphics[scale=0.145]{img/irregular_shape/reference_test_symmetric/tumor_DG-UPW_Pi1_n_i-500_cropped.png}} \\
% 			\rotatebox[origin=c]{90}{\textbf{Non-symmetric}} &
% 			\raisebox{-0.47\height}{\includegraphics[scale=0.145]{img/irregular_shape/reference_test/tumor_DG-UPW_Pi1_n_i-100_cropped.png}} &
% 			\raisebox{-0.47\height}{\includegraphics[scale=0.145]{img/irregular_shape/reference_test/tumor_DG-UPW_Pi1_n_i-200_cropped.png}} &
% 			\raisebox{-0.47\height}{\includegraphics[scale=0.145]{img/irregular_shape/reference_test/tumor_DG-UPW_Pi1_n_i-500_cropped.png}}
% 		\end{tabular}
% 		\caption{Nutrients.}
% 	\end{figure}
% \end{frame}
\begin{frame}{Irregular tumor growth {\footnotesize (more tests in \cite{acosta2023structure})}}
	\scriptsize
	\begin{figure}
		\centering
		\hspace*{-0.6cm}
		\begin{tabular}{cc}
			\hspace*{-1cm} Symmetric & \hspace*{-1cm} Non-symmetric \\
			\animategraphics[autoplay,loop,width=5cm]{5}{img/animation/irregular_shape/reference_test_symmetric/nutrients/tumor_DG-UPW_Pi1_n_i_cropped-}{0}{50} &
			\animategraphics[autoplay,loop,width=5cm]{5}{img/animation/irregular_shape/reference_test/nutrients/tumor_DG-UPW_Pi1_n_i_cropped-}{0}{50}
			% \includemovie[repeat,autoplay]{5cm}{4.07cm}{img/irregular_shape/reference_test_symmetric/tumor_DG-UPW_Pi1_n_reduced.mp4} &
			% \includemovie[repeat,autoplay]{5cm}{4.07cm}{img/irregular_shape/reference_test/tumor_DG-UPW_Pi1_n_reduced.mp4}
		\end{tabular}
		\caption{Nutrients.}
	\end{figure}
\end{frame}

\section{Future work}

\begin{frame}{Future work}
	\begin{itemize}\itemsep1em
		\item Adapting the model to fit real data.
		\item Coupling this model with fluid or porous media equations.
		\item Carrying out an error analysis of the method.
		\item Exploring other options to avoid restrictions on the mesh.
		\item Extending this method to higher-order approximating polynomials.
	\end{itemize}
\end{frame}

\begin{frame}{References}
	\scriptsize 
	\vspace*{-0.25cm}
	\nocite*
	\bibliographystyle{apalike}
	\bibliography{references}
\end{frame}

\begin{frame}{}
	\centering
	\vspace*{1cm}
	{\Huge
		\emph{Thanks for your attention!}}
	
	\vspace*{1cm}
	\begin{acknowledgements}
		The speaker has been supported by a \textit{Graduate Scholarship funded by the University of Tennessee at Chattanooga}; by \textit{UCA FPU contract UCA/REC14VPCT/2020, Erasmus+ KA131 and travel grants funded by Universidad de Cádiz}.
		
		The collaborators have been supported by \textit{Grant US-4931381261 (US/JUNTA/FEDER, UE)}.
	\end{acknowledgements}
\end{frame}